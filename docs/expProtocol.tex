\documentclass[a4paper]{article}
\usepackage[english]{babel}
\usepackage[latin1]{inputenc}
\usepackage[pdftex]{hyperref}
\usepackage{graphicx}
\hypersetup{pdftitle={Fatigue experiments with Lumikki}}
\usepackage{SIunits}
\usepackage{color}
\usepackage{framed}

\begin{document}

\begin{figure}[htb]
  \begin{minipage}[\textwidth]{\textwidth}
    {\color{red}
     {\Large IMPORTANT!}

     {\Large Operator safety}
     \begin{itemize}
       \item Do not place your hand in between the crosshead and the compression plate unless the measurement device is on \textsf{MODE 0}! To operate the clamps, use manual switches located in the clamp body (operator panel does not work in 0 mode).
       \item In case the machine must be stopped immediately, press the red button on the control panel. To release the button, rotate it clockwise.
       \item Upper safetylimit is 10 mm for force controlled pulling experiment (e.g. paper creep). It takes more than 10 mm for clamps to stop moving.
     \end{itemize}

     {\Large General user rules}
     \begin{itemize}
       \item Log out after your experiments for all machines.
       \item If somebody is logged in, you are not allowed kick him out. For example, if you are trying to contact the computer remotely, click "No", when remote desktop asks: "User XYZ is logged in and his session will be closed. Do you want to proceed?"
     \end{itemize}
     }
  \end{minipage}
\end{figure}

\clearpage

\section{Foreword}

This is a list of instructions to bring the Instron measurement device
and necessary camera gear on-line and to perform wood fatigue
tests. Please read the manual through before attempting to use the
equipment. The Instron measurement device contain safety mechanisms to
stop the user from harming it. Nonetheless, care should be taken not
to drive the crosshead into the sample plate. This can cause damage to
the force gauge (rather expensive) in the device.

\section{Before you start doing set of experiments}

\begin{itemize}
  \item Reboot windows (clear memory).
  \item Reboot instron (clear memory).
\end{itemize}

\section{Changing the clamps}

There are more accurate instructions next to the machine, but here is some repetition. 

\subsection{Basic issues}

\begin{itemize}
  \item ALWAYS MAKE SURE the machine is in \textsf{MODE 0} before you put your hand between the clamps!!
  \item Put the used screws etc. into the right boxes. Note that some parts are used in both clamp types.
  \item You can roughly adjust the distance between the upper and lower part as follows:
  \begin{itemize}
    \item Set the machine on \textsf{MODE 0}.
    \item Roll the black handles on the upper part of the machine open. If the author's memory serves her right, it opens by rotating counter-clockwise.
    \item Move the upper part of machine by pressing the black switch close to the manual control buttons.
    \item Close the black handles tightly as you're done.
  \end{itemize}
  \item Put the tools into their places as you have changed the clamps!
\end{itemize}

\subsection{Removing the compression plates}

\begin{itemize}
  \item Remove the three screws of the bottom plate and remove the plate.
  \item Remove the screw from the bottom of the structure you recently removed and pull the parts apart.
  \item Use two wrenches to loosen the tighting part (silvery) at the upper plate.
  \item Rotate the upper plate in order to remove it. BE CAREFUL, the plate is some heavy and it drops suddenly as the screw is loose enough!
  \item Use hex head wrench to remove the special screw from the upper plate.
  \item If you are going to change these plates to the stretching grips, save all the other parts except the compression plates. Put compression plates to their box.
\end{itemize}

\subsection{Removing the stretching grips}

\begin{itemize}
  \item Remove the pneumatic tubes: take a grip on the ring around the end of the tube, push and rotate it.
  \item Remove the three screws of the bottom grip and remove the grip.
  \item Remove the screw from the bottom of the structure you recently removed and pull the parts apart. (They are really tight.)
  \item Use two wrenches to loosen the tighting part (silvery) at the upper grip.
  \item Rotate the upper grip in order to remove it. BE CAREFUL, the grip is quite heavy and it drops suddenly as the screw is loose enough!
  \item Use hex head wrench to remove the special screw from the upper grip.
  \item If you are going to change these grips to the compression plates, save all the other parts except the stretching grips. Put stretching grips to their box.
\end{itemize}

\subsection{Installing the compression plates}

\begin{itemize}
  \item Take a compression plate and the round bottom part with six screw holes. Put these parts together with a screw.
  \item Attach the bottom part onto the bottom of the machine by three screws.
  \item Take the special screw (the black long one that lacks proper head) and another compression plate. Put the screw into the plate.
  \item Take the tightening part and put it around the screw. Screw the structure onto the upper part of the machine.
  \item Tighten the tightening part by two wrenches.
  \item Adjust the height of the upper plate (see earlier). Then, put the machine on \textsf{MODE II} and, by using arrow buttons, move the plates very close to each others (not touching).
  \item Use Instron program to calibrate the machine.
\end{itemize}

\subsection{Installing the stretching grips}

There is slightly more details in order to put the grips straight.

\begin{itemize}
  \item Take a stretching grip and the round bottom part with six screw holes. Set the parts together in the way that the stretching grip is as horizontal compared to the camera as possible. Put these parts together with a screw.
  \item Attach the bottom part onto the bottom of the machine by three screws.
  \item Take the special screw (the black long one that lacks proper head) and another stretching grip. Put the screw into the grip.
  \item Take the tightening part and put it around the screw. Screw the structure onto the upper part of the machine, but not overly tightly.
  \item Attach the pneumatic tubes onto the grips (push them hard onto the grip by holding the ring around the end of the tube until you hear a click).
  \item Take a metal plate and put it into lower grip. Close the grip.
  \item Lower the upper grip enough to be able to grab the upper part of the metal plate. Close the upper grip. Now, the grips should be straight compared to each others.
  \item Tighten the tightening part by two wrenches.
  \item Open the grips and remove the metal plate.
\end{itemize}

\section{Before starting each experiment}

\begin{itemize}
  \item Be sure that the Instron machine is calibrated. For example, in constant strain compression experiments, the force indicator on Instron Computer (second computer from left) should give 0.000 kN in the start.
  \item If needed, calibrate the machine (adviced later in Troubleshooting section).
\end{itemize}

\section{Changing variables}

\begin{itemize}
  \item Variables can be chanced in tabs in Lumikki.
  \item Remember! Every time you have chanced the variables, update them to Lumikki by clicking ''Apply'' at the end of the variable list!
\end{itemize}

\subsection{Creep test}

\begin{itemize}
  \item Go to the TTM tab in Lumikki and change the \textit{protocol name} variable to ''creep'' (without quotas).
  \item The only other parameter to change is \textbf{load}.
  \item If you are doing a compression tests, give negative values to the load. If you are doing stretching tests, give positive values.
\end{itemize}

\subsection{Tensile test}

\begin{itemize}
  \item Go to the TTM tab in Lumikki and change the \textit{protocol name} variable to ''tensile'' (without quotas).
  \item Tensile tests are supposed to follow ISO 527-1 standard. This means that the \textbf{ramp rates} of the tests should be some of the following:
  \begin{itemize}
    \item 0.0333 mm/s (aka 2 mm/min, like it stands officially in the standard)
    \item 0.0833 mm/s (aka 5 mm/min)
    \item 0.1667 mm/s (aka 10 mm/min)
    \item 0.3333 mm/s (aka 20 mm/min)
    \item 0.8333 mm/s (aka 50 mm/min)
    \item 1.6667 mm/s (aka 100 mm/min)
    \item 3.3333 mm/s (aka 200 mm/min)
    \item 8.3333 mm/s (aka 500 mm/min)
  \end{itemize}
  \item In the standard, there is also 1 mm/min, but it is too slow for our machine.
  \item Make sure that the ramp amplitude is large enough: for example, with the wood samples, it should be bigger than the height of the sample to make sure the machine really compresses it enough.
  \item If you are doing a compression tests, give negative values to the ramp rate and the ramp amplitude. If you are doing stretching tests, give positive values.
\end{itemize}

\section{Checking that the camera takes good images}

\begin{itemize}
  \item Go to the Camera Computer (first computer from right) and open program called Measurement \& Automation Explorer.
  \item Choose from side bar menu: My System $\rightarrow$ Devices and Interfaces $\rightarrow$ NI-IMAQdx Devices $\rightarrow$ cam0: DALSA Genie HM1024.
  \item Then, click button "Grab". Now, you should see a real-time video image of the clamps/compression plates of the machine.
  \item With the ''video'', you can check if the camera takes proper images, the sample piece is set straight, etc.
\end{itemize}

\section{Putting the sample to the machine}

\begin{itemize}
  \item Set the machine on \textsf{MODE I}, and move the clamps enough for setting the sample piece right.
  \item Then, \textit{set the machine on} \textsf{MODE 0}, to be sure that the clamps/compression plates are not moving (onto your fingers) when you operate with the sample.
  \item Put the sample to the clamps/on the compression plate and check that the camera image looks clear. Adjust lights, camera lenses, position of the sample etc. if needed.
  \item After all looks clear, switch the machine again on \textsf{MODE II}.
\end{itemize}

In some wood compression experiments, you can also put around the sample 
black cardboard pieces (next to the machine or in the shelf in the neighboring 
room) to avoid the pieces of explosively shattered samples from flying all 
over the laboratory. The samples rarely break down in that way, but it is 
easier to collect the pieces if it happens. Samples with vertical annual ring 
position are in greatest risk to do this.

\section{Initial straightening/pinching of the sample} 

\begin{itemize}
  \item Go to the Instron Computer and check that the Specimen Protect button is on. The button can be recognized by shield image that turns green when Specimen Protect is on.
  \item Then, go to the machine and move the clamps/plates in the way that either the sample paper gets straightened or the sample wood piece gets pinched between the compression plates. Specimen Protect makes sure that the sample doesn't get ripped or compressed too much, but the sample is just adjusted to stay in place.
\end{itemize}

\section{Just before starting the experiment itself} 

\begin{itemize}
  \item \textit{Turn off the Specimen Protect from the Instron Computer AND the Measurement \& Automation Explorer from the Camera Computer.} If the latter one asks about saving, click "No".
  \item Then, go to Lumikki Computer (first computer from left) and open its control tab from web browser. Name your experiment to the text field of the front page.
  \item Then, \textit{click button "Force status"}, to make sure that all parts of the experiment machine are ready to measure. 
\end{itemize}

\section{Starting the experiment} 

\begin{itemize}
  \item Once the "Force status" button tells that the whole experiment machine is ready for measurement, click "Run".
  \item Now the experiment starts, and you can follow the stress-strain curve from the Instron Computer, in LabVIEW  window called startLogging.poller.
  \item As the load/height limit trips, the machine tells you about that. Click ''Stop'' in Lumikki.
\end{itemize}

\section{After experiments}

\begin{itemize}
  \item Make sure that the machine isn't gathering data anymore. In other words, click button "Stop" after each experiment, especially the last one. Otherwise, the experiment setting continues gathering data after you have left, with may eventually fulfill the data storage of the computers.
  \item Put the machine to (analog) position control.
\end{itemize}

\clearpage

\section{Instron initialization}

First, before anything else is done with the measurement device, it
must be initialized properly. In the case of compression tests, the
measurement device must be configured so that the crosshead does not
reach the sample plate when fully extended. Also setting proper safety
limits and calibration are necessary.


To begin turn on the measurement device. The power switch is on the
back side of the gray box next to the concrete table. The device boots
up for a while and remains in \textsf{MODE 0}, untill at some point it
can be switched to \textsf{MODE I}. The device itself does not give
any signal about this. but if you watch the log on the TTM machine,
you will notice activity when the measurement device is ready to be
switched to \textsf{MODE I}.

While the machine is booting up, check logs from the {\it Instron
control program} to see when the machine was used last. If you are
confident that the machine was last used for the same measurement and
has been initialized already, you can skip the rest of this section
and set the machine on \textsf{MODE I}.


\begin{enumerate}
  \item Set machine to \textsf{MODE I}.
  \begin{itemize}
    \item Clear all safety limits.
    \item Loosen the two black screw bolts. When open, the handles are to be set downwards (as if they were valves). When closed, the handles point inwards or toward each other.
    \item Lift the machine using the black rocker switch enough so that you can fully extend the crosshead.
    \item Using the arrow buttons on the remote, extend the driver head fully taking caution not to allow the driver head to touch the sample pedestal.
    \item Using the black rocking switch, bring down the machine slowly until there is only a small gap (about $0.5$ \milli\meter) between the driver head and the compression plate. {\bf Take care not to ram the driver head into the compression plate!}
    \item Secure the machine by tightening the screw bolts. Leave the handles pointing towards each other.
  \end{itemize}
  \item Switch the machine to \textsf{MODE II}.
  \begin{itemize}  
    \item Calibrate the force sensor by opening the force tab, clicking ``Calibration'' and then ``Balance''.
    \item Set force safety limits to $\mathbf{\pm 1000}$ {\bf\newton}.
    \item Set digital position to zero by opening digital position tab, clicking ``Calibration'' and then ``Balance''.
    \item Set position safety limits to $\mathbf{\pm 30}$ {\bf\milli\meter}. To do this, you must set Instron to \textsf{MODE 1} and moove the crosshead up a bit, as the crosshead is bellow the $\mathbf{-30}$ {\bf\milli\meter} safety limit at this point.
  \end{itemize}
\end{enumerate}

\clearpage

\section{FAQ}

\subsection{How can I calibrate the force and position meters?}

\begin{itemize}  
  \item Open the Instron panel and choose either Position or Load tab, depending on which you want to calibrate.
  \item Here, open Primary Limits section from side menu, and take the ticks away from parts "Limit enabled" and "Limit tripped".
  \item Turn Specimen Protect off.
  \item Open Calibration section and click "Balance".
  \item Go back to Primary Limits section and put ticks back to "Limit enabled" and "Limit tripped". If needed, turn Specimen Protect on.
\end{itemize}

\subsection{I have a problem with the Instron Computer, how can I restart it?}

\begin{itemize}  
  \item Close all programs.
  \item Restart the computer normally.
  \item Open the Instron Console and wait that its panel stops flashing or turns green.
  \item Open LabVIEW from Start menu and choose there lumikkiInterface. Open and activate all windows that end with ".poller".
  \item Go the the Instron Console panel, then its tabs Position and Load. Go to Primary Limits section and put ticks to "Limit enabled" and "Limit tripped" in both tabs.
\end{itemize}

\subsection{I accidentally forgot to click "Stop" after my last experiment, and the computers continued data unnecessary gathering all the night. What should I do?}

\begin{itemize}  
  \item Delete the experimental stress/strain data of the over-long experiment from the Instron Computer and corresponding pictures from the Camera Computer to free memory.
  \item Restart the Instron Computer and open necessary LabVIEW programs (see the previous question).
\end{itemize}
\end{document}
